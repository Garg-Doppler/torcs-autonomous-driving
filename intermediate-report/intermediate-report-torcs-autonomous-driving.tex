%
% Example for a student report latex file. Adapt as necessary
%

% the following lines should stay as is
\documentclass[10pt,a4paper,twoside,journal]{IEEEtran}
\usepackage[nocompress]{cite}
\usepackage[pdftex]{graphicx}

% some packages that most people use or improve the result
\usepackage[utf8]{inputenc}
\usepackage[english]{babel}
\usepackage{amsmath}
\interdisplaylinepenalty=2500
\usepackage{amssymb}
\usepackage{amsfonts}
\usepackage{amsbsy}
\usepackage{flushend}
% properly print units, enable compact product between units
\usepackage[inter-unit-product =\cdot]{siunitx}
% load units \bit, \byte usw
\sisetup{detect-weight=true, binary-units=true}
\usepackage{array}
\usepackage{xspace}
\usepackage{algorithm}
\usepackage{url}
\usepackage[pagebackref=true,breaklinks=true,colorlinks,bookmarks=false]{hyperref}

\usepackage[switch]{lineno}
\makeatletter
\@ifpackageloaded{lineno}{%
	\newcommand*\patchAmsMathEnvironmentForLineno[1]{%
	  \expandafter\let\csname old#1\expandafter\endcsname\csname #1\endcsname
	  \expandafter\let\csname oldend#1\expandafter\endcsname\csname end#1\endcsname
	  \renewenvironment{#1}%
		 {\linenomath\csname old#1\endcsname}%
		 {\csname oldend#1\endcsname\endlinenomath}}%
	\newcommand*\patchBothAmsMathEnvironmentsForLineno[1]{%
	  \patchAmsMathEnvironmentForLineno{#1}%
	  \patchAmsMathEnvironmentForLineno{#1*}}%
	\AtBeginDocument{%
	\patchBothAmsMathEnvironmentsForLineno{equation}%
	\patchBothAmsMathEnvironmentsForLineno{align}%
	\patchBothAmsMathEnvironmentsForLineno{flalign}%
	\patchBothAmsMathEnvironmentsForLineno{alignat}%
	\patchBothAmsMathEnvironmentsForLineno{gather}%
	\patchBothAmsMathEnvironmentsForLineno{multline}%
	}}%
	{}
\makeatother

% add your additional packages here
\usepackage{lipsum}

\begin{document}

%
% When submitting the review, make sure to include the following line to enable
% line numbering. When submitting the final report, disable the following line!
\linenumbers
%

%
% configure submission details
%

% here you can specify the day of submission
\newcommand{\submissiondate}{\today}

% please specify the type of your submission. E.g. Advanced Seminar or Practical
% Laboratory
\newcommand{\submissiontype}{Practical course: Computational Neuro Engineering}

% give information about when your course happened in form of SEMESTER YEAR,
% e.g. Winter Semester 2016. In addition, specify the "short version", e.g. WS
% 2016
\newcommand{\submissionterm}{Winter Semester 2017/2018}
\newcommand{\submissiontermshort}{WS 2017/2018}

% the full submission title
\newcommand{\submissiontitle}{Learning to drive based on multiple sensor cues
in The Open Racing Car Simulator (TORCS)}
% in case that you have a very long report title, make sure to provide a shorter
% version that can be used in the \markboth command further below. In case your
% topic is short, simply comment the next and uncommented the second line
\newcommand{\submissiontitleshort}{Autonomous driving in TORCS with multiple sensors}
%\newcommand{\submissiontitleshort}{\submissiontitle}

% author list. Make sure that you include your matriculation number in the
% section starting with \thanks. In addition, specify your supervisor(s).
\author{Tim~Bicker and Nimar~Blume%
	\thanks{\textbf{Authors}:
		Tim Bicker (12345678, tim.bicker@tum.de),
		and Nimar Blume (03638934, nimar.blume@tum.de)
		% the following includes additional information
		\textbf{Course}: \submissiontype{} \submissionterm{}
		\textbf{Submitted}: \submissiondate{}
		\textbf{Supervisor}: Florian Mirus.
		Neuroscientific System Theory (Prof. Dr. J\"org Conradt), Technische
		Universit\"at M\"unchen, Arcisstraße 21, 80333 M\"unchen, Germany.
}}

% both items should look alike and contain a short version of the type of work
% and semester (e.g. Advanced Seminar WS 2016, Project Laboratory SS 2017) and
% your report title. If your title is too long, find a shorter one.
\markboth{\submissiontype{} \submissiontermshort{}: \submissiontitleshort{}}
{\submissiontype{} \submissiontermshort{}: \submissiontitleshort{}}

% this generates the paper title
\title{\submissiontitle}
\maketitle

% write a short abstract to introduce the reader to your work
\begin{abstract}
	To implement an autonomous driver in The Open Racing Car Simulator (TORCS)
	based on a deep neural network (DNN) and spiking neural network (SNN) multiple
	sensor cues are used. Specifically, the DNN predicts the current car displacement
	and angle to the road centre from a driver's view image. Based on the two values
	a SNN generates driving commands for the car. Subsequently, the car is put onto a 
	new track and the driving performance is evaluated.\
	The DNN is based on a Convolutional Neural Network and after training the mean 
	absolute error for the displacement is XXXX and for the angle is XXX on an
	unseen test track.
\end{abstract}

\begin{IEEEkeywords}
	deep learning, TORCS, convolutional neural network, spiking neural network, autonomous driving
\end{IEEEkeywords}

%
% Main body follows here. Only capitalize the first word in any title
%
\section{Introduction}
\label{sc:intro}

\IEEEPARstart{W}{riting} a report is almost straightforward as long as you are
properly prepared. Make sure that you found a suitable structure for your
report. For instance, you could split up a report into the following sections:

However you may have to adapt to the report you are writing. An advanced seminar
usually does not contain experiments and evaluation, but you wish to show
details about one references paper that you found.

If you're unsure how to structure your report, contact your supervisor. Together
with your supervisor you should be able to figure out how to best organize your
manuscript.

Then the remaining task mostly consists of writing. Writing a good paper only
comes with experience, though, and an understanding how a good paper should be
structures. An excellent overview about the latter issue can be found in
\cite{katzoff1964}.

\section{State of the art}
\label{sc:sota}

You are free to use any additional packages that you require. There are some
caveats, though. Make sure to include packages that do not ship with standard
\LaTeX{} distributions in your final submission. Most of the important packages
are included already in the preamble of this document, though.


\section{Implementation details}
\subsection{DNN: Keras}
\label{ssc:keras}
\subsection{SNN: Nengo}
\label{ssc:nengo}

Before handing in your thesis, even for an intermediate review, please perform a
spellcheck and correct grammar mistakes. The report is not meant to be a
narrative text. Please stick to neutral and technical style and avoid subjective
or biased expressions or adjectives/adverbs such as \emph{obviously, always,
very, especially well, actually, so-called etc}. Scientific writing is about
precision and you should underpin your statements factually, not soften them
with unnecessary qualifiers.

Sometimes you have a wide figure or environment. Inclusion can be achieved using
the \texttt{figure*} environment as used in Figure \ref{fig:twocolumnfigure}.
\begin{figure*}
	\centering
	\fbox{\rule{0pt}{2cm} \rule{1.0\linewidth}{0pt}}
	\caption{A wide figure.}
	\label{fig:twocolumnfigure}
\end{figure*}

\section{Experiments and evaluation}
\label{sc:evaluation}

In the following we will give you some short information about how you should
typeset mathematical equations and set tables or figures. If you require more
information about the topics then first try to find answers on the Internet, and
only if you did not figure out how to solve your issue, contact your supervisor.

\subsection{Mathematical equations an}

In case you have to typeset equations make sure that all equations are numbered.
The following example shows how to do so

\begin{equation}
	E = mc^2
\end{equation}

Sometimes you wish to align equations. This is possible with the (already
included) \texttt{align} package. The example in Equation \ref{eq:lotkavolterra}
which gives the Lotka--Volterra, or predator-prey equations, shows how to use
it.

\begin{align}\label{eq:lotkavolterra}
	\frac{dx}{dt} &= \alpha x - \beta x y \\
	\frac{dy}{dt} &= \delta x y - \gamma y
\end{align}

\subsection{Figures, tables, algorithms}

Most reports need to include one of the mentioned objects. There are suitable
environments for each of them. If you are interested in typesetting algorithms,
have a look at the packages \texttt{algorithmic} and \texttt{algorithmx}. Make
sure that all algorithms are well formatted and clearly understandable!

Figures are included using the \texttt{figure} environment. All graphics should
be centered.  Please ensure that any point you wish to make is resolvable in a
printed copy of the paper. Resize fonts in figures to match the font in the
body text, and choose line widths which render effectively in print. An example
is shown in Figure \ref{fig:onecolumnfigure}.
\texttt{includegraphics} will include the respective file.
\begin{figure}
	\centering
	\fbox{\rule{0pt}{2cm} \rule{1.0\linewidth}{0pt}}
	\caption{Always add a short caption to your figures.}
	\label{fig:onecolumnfigure}
\end{figure}


\section{Conclusion}
\label{sc:conclusion}

When submitting your final report, make sure to include all files that are
required to build the PDF from source. This includes all bibliography files and
\LaTeX{} files. In addition, don't forget to include all figures. If you
performed an evaluation on a dataset you should include this data as well. For
project laboratories or seminars that produced source code, this should be
shipped with your final submission as well. If you prefer to keep your code
secret, contact your supervisor.

%
% If you wish to put your work under a specific license, your free to do so. You
% are not obliged, so you can remove the following section if you wish to.
%
\section*{License}
\markright{LICENSE}
This work is licensed under the Creative Commons Attribution 3.0 Germany
License. To view a copy of this license,
visit \href{http://creativecommons.org/licenses/by/3.0/de/}{http://creativecommons.org} or send a letter
to Creative Commons, 171 Second Street, Suite 300, San
Francisco, California 94105, USA.


% each report should include all references that you cite in the work. Make sure
% that you include all references!
\bibliographystyle{ieee}
\bibliography{bibliography}

\end{document}
